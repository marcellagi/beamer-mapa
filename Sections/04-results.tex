% %*----------- SLIDE -------------------------------------------------------------
% \begin{frame}[t]{Finalização}
%     \begin{itemize}
%         \item Cada líder deverá realizar a apresentação final do desafio no dia 25/maio/2020.
%         \item No dia da apresentação, somente o líder poderá responder os questionamentos emitidos pelos facilitadores.
%         \item A avaliação será da equipe, não havendo avaliação individual dos integrantes da equipe com exceção do líder de cada equipe.
%         \item A apresentação deverá ser desenvolvida em latex.
%         \item Os videos dos desafios deverão estar contidos na apresentação final.
%         \item Os videos deverão ser completos, tendo começo, meio e fim da missão realizada.
%     \end{itemize}
% %*----------- notes
%     \note[item]{Notes can help you to remember important information. Turn on the notes option.}
% \end{frame}
% %-
% %*----------- SLIDE -------------------------------------------------------------
% \begin{frame}[c]{A importância atual da robótica}
%     \begin{center}
    
%         \includemedia[
%             width=0.7\linewidth,
%             totalheight=0.39375\linewidth,
%             activate=pageopen,
%             passcontext, 
%             %transparent,
%             addresource=./Source/gifs/robotdesinfec.wmv,
%             flashvars={
%             source=./Source/gifs/robotdesinfec.wmv
%             &autoPlay=true
%             &autoRewind=true
%             &loop=true}
%             ]{\fbox{\includegraphics{Source/gifs/robotdesinfec0.png}}}{VPlayer.swf}
%     \end{center}

% %*----------- notes
%     \note[item]{Notes can help you to remember important information. Turn on the notes option.}
%  \end{frame}
% %-
% %*----------- SLIDE -------------------------------------------------------------
% \begin{frame}[fragile]{A importância atual da robótica}
%     Para a implentação de R gráficos deve-se realizar os seguintes comando no ambiente R:

%     \begin{lstlisting}[language=R]
%         library(tikzDevice)
%         beamer.parms = list(paperwidth   = 364.19536/72,
%                     paperheight  = 273.14662/72,
%                     textwidth    = 307.28987/72,
%                     textheight   = 269.14662/72)
%         tikz("./your_file.tex", 
%             width = beamer.parms$textwidth, 
%             height = beamer.parms$textheight)
%         ggqqplot(na.omit(my_data$col2))
%         dev.off()
%     \end{lstlisting}

%     \begin{columns}
%         \column{.01\textwidth}
%         \column{.59\textwidth}
%             \textbf{A penúltima linha do texto acima é o códifo em R para a construção do gráfico.}
                
%         \column{.4\textwidth}
%             \centering
%             \begin{tikzpicture}[thick, scale=0.4, every node/.style={scale=0.1}]
%                 \node[at=(current page.center)] {
%                %\input{./Media/r-graphics/img-marco1.tex}
%                 \input{./Source/r-graphics/graficox.tex}
%                 };
%             \end{tikzpicture}
%     \end{columns}
% %*----------- notes
%     \note[item]{Notes can help you to remember important information. Turn on the notes option.}
%  \end{frame}
% %-
% %*----------- SLIDE -------------------------------------------------------------
% \begin{frame}[c]{A importância atual da robótica}
%     \centering
%     \framesubtitle{robo}
%     \begin{bclogo}[ 
%         couleur=white!10!white,
%         ombre=false,
%         epBord=3,
%         couleurBord = gcolor,
%         arrondi = 0.2,
%         logo=\bcinfo]{}
%         \centering
%         \begin{tikzpicture}[thick, scale=0.35, every node/.style={scale=0.5}]
%             \node[at=(current page.center)] {
%             \input{./Source/r-graphics/graficox.tex}
%             };
%         \end{tikzpicture}
%     \end{bclogo}
% %*----------- notes
%     \note[item]{Notes can help you to remember important information. Turn on the notes option.}
%  \end{frame}
%  %-
% %*----------- SLIDE -------------------------------------------------------------
% \begin{frame}
%     \begin{center}
%         \vspace*{1.5cm}
%         \textbf{\Huge{\textcolor{mracula5}{MUDANÇA}}}
%     \end{center}
    
% %*----------- notes
%     \note[item]{Notes can help you to remember important information. Turn on the notes option.}
%  \end{frame}
%  %-
% %*----------- SLIDE -------------------------------------------------------------
% \begin{frame}
%     %\transdissolve[duration=0.5]
%     %\hspace*{-1cm}
%     \begin{columns}
%         %\column{.01\textwidth}
%         \column{0.4\textwidth}
%             ~\hfill
%             \vbox{}\vskip-1.4ex%
%             \begin{beamercolorbox}[sep=8em, colsep*=18pt, center, wd=\textwidth, ht=\paperheight]{title page header}%
%                 \begin{center}
%                     \textbf{\huge{VISÃO}}\par
%                     \vspace*{0.3cm}
%                     \textbf{\huge{DO}}\par
%                     \vspace*{0.3cm}
%                     \textbf{\huge{FUTURA}}
%                 \end{center}
%             \end{beamercolorbox}%
%         \hfill\hfill
%         \column{.05\textwidth} 
%         \column{.6\textwidth}
%             \small{\lipsum[2-1]}
%             \begin{itemize}
%                 \item tópico 1
%                 \item tópico 2 
%                 \item \xout{tópico 3}
%                 \item \sout{last tópico}
%             \end{itemize}
%     \end{columns}
  
%  %*----------- notes__
%     \note[item]{Notes can help you to remember important information. Turn on the notes option.}
% \end{frame}
% %-
% %*----------- SLIDE -------------------------------------------------------------
% \begin{frame}
%     %\transdissolve[duration=0.5]
%     %\hspace*{-1cm}
%     \begin{columns}
%         %\column{.01\textwidth}
%         \column{0.4\textwidth}
%        %~\hfill
%         %\begin{beamercolorbox}[sep=8em, colsep*=18pt, center, wd=\textwidth, ht=\paperheight]{title page header}%
%         %\vspace*{-1.5cm}%
%         % \raggedright
%         % \begin{figure}[p]
%         %     \includegraphics[trim = 0 0 0 10, clip, width=1.03\paperwidth, height=1.03\paperheight, keepaspectratio=true]{melody-p-wFN9B3s_iik-unsplash.jpg}
%         % \end{figure}

%         \begin{tikzpicture}[remember picture,overlay]
%             \node [xshift=3cm,yshift=-4.1cm] at (current page.north west) {
%                 \includegraphics[width=\paperwidth, height=\paperheight, keepaspectratio]{melody-p-wFN9B3s_iik-unsplash.jpg}
%             };
%         \end{tikzpicture}
%         %\end{beamercolorbox}%
%         %\hfill\hfill
%         \column{.05\textwidth} 
%         \column{.6\textwidth}
%             \small{\lipsum[2-1]}
%             \begin{itemize}
%                 \item tópico 1
%                 \item tópico 2 
%                 \item \xout{tópico 3}
%                 \item \sout{last tópico}
%             \end{itemize}
%     \end{columns}
  
%  %*----------- notes__
%     \note[item]{Notes can help you to remember important information. Turn on the notes option.}
% \end{frame}
% %-
% %*----------- SLIDE -------------------------------------------------------------
% \begin{frame}
%     %\transdissolve[duration=0.5]
%     %\hspace*{-1cm}
%     \begin{columns}
%         %\column{.01\textwidth}
%         \hspace*{0.5cm}
%         \column{.6\textwidth}
%             \small{\lipsum[2-1]}
%             \begin{itemize}
%                 \item tópico 1
%                 \item tópico 2 
%                 \item \xout{tópico 3}
%                 \item \sout{last tópico}
%             \end{itemize}
%         \column{.05\textwidth} 
%         \column{0.4\textwidth}
%             ~\hfill
%             \vbox{}\vskip-1.4ex%
%             \begin{beamercolorbox}[sep=8em, colsep*=18pt, center, wd=\textwidth, ht=\paperheight]{title page header}%
%                 \begin{center}
%                     \textbf{\huge{VISÃO}}\par
%                     \vspace*{0.3cm}
%                     \textbf{\huge{FUTURA}}
%                 \end{center}
%             \end{beamercolorbox}%
%         \hfill\hfill
%     \end{columns}
  
%  %*----------- notes__
%     \note[item]{Notes can help you to remember important information. Turn on the notes option.}
% \end{frame}
% %-
% %*----------- SLIDE -------------------------------------------------------------
% \begin{frame}
%     %\transdissolve[duration=0.5]
%     %\hspace*{-1cm}
    
%     \begin{columns}
%         %\column{.01\textwidth}
%         \column{0.4\textwidth}
%             ~\hfill
%             \vbox{}\vskip-1.4ex%
%             \begin{beamercolorbox}[sep=8em, colsep*=18pt, wd=\textwidth,ht=\paperheight]{title page header}
%                 \begin{center}
%                     \textbf{\huge{VISÃO}}\par
%                     \vspace*{0.3cm}
%                     \textbf{\huge{FUTURA}}
%                 \end{center}
%             \end{beamercolorbox}%
%         \column{.05\textwidth} 
%         \column{.6\textwidth}
%         \begin{center}
%             \begin{figure}
%                 \includegraphics[width=.5\textwidth]{darwin-op-2}
%                 \caption{Darwim OP \cite{webcite}}
%             \end{figure}
            
%         \end{center}
            
%     \end{columns}
  
%  %*----------- notes
%     \note[item]{Notes can help you to remember important information. Turn on the notes option.}
% \end{frame}
%  %-
% %*----------- SLIDE -------------------------------------------------------------
% \begin{frame}
%     %\transdissolve[duration=0.5]
%     %\hspace*{-1cm}
    
%     \begin{columns}
%         %\column{.01\textwidth}
%         \column{.6\textwidth}
%             \begin{center}
%                 \begin{figure}
%                     \includegraphics[width=.5\textwidth]{darwin-op-2}
%                     \caption{Darwim OP \cite{webcite}}
%                 \end{figure}
                
%             \end{center}
%         \column{.05\textwidth}
%         \column{0.4\textwidth}
%             ~\hfill
%             \vbox{}\vskip-1.4ex%
%             \begin{beamercolorbox}[sep=8em, colsep*=18pt, wd=\textwidth,ht=\paperheight]{title page header}
%                 \begin{center}
%                     \textbf{\huge{VISÃO}}\par
%                     \vspace*{0.3cm}
%                     \textbf{\huge{FUTURA}}
%                 \end{center}
%             \end{beamercolorbox}%
%     \end{columns}
  
%  %*----------- notes
%     \note[item]{Notes can help you to remember important information. Turn on the notes option.}
% \end{frame}
% %-
% %*----------- SLIDE -------------------------------------------------------------
% % \begin{frame}
% %     %\transdissolve[duration=0.5]
% %     %\hspace*{-1cm}
    
% %     \begin{columns}
% %         %\column{.01\textwidth}
% %         \column{.6\textwidth}
% %             \begin{tikzpicture}[node distance=1.cm, every node/.style={fill=white, font=\sffamily}, align=center]
% %             % Specification of nodes (position, etc.)
% %                 \node (start)             [activityStarts]              {Activity starts};
% %                 \node (onCreateBlock)     [process, below of=start]          {onCreate()};
% %                 \node (onStartBlock)      [process, below of=onCreateBlock]   {onStart()};
% %                 \node (onResumeBlock)     [process, below of=onStartBlock]   {onResume()};
% %                 \node (activityRuns)      [activityRuns, below of=onResumeBlock] {Activity is running};
% %                 \node (onPauseBlock)      [process, below of=activityRuns, yshift=-1cm] {onPause()};
% %                 \node (onStopBlock)       [process, below of=onPauseBlock, yshift=-1cm] {onStop()};
% %                 \node (onDestroyBlock)    [process, below of=onStopBlock, yshift=-1cm] {onDestroy()};
% %                 \node (onRestartBlock)    [process, right of=onStartBlock, xshift=4cm] {onRestart()};
% %                 \node (ActivityEnds)      [startstop, left of=activityRuns, xshift=-4cm] {Process is killed};
% %                 \node (ActivityDestroyed) [startstop, below of=onDestroyBlock] {Activity is shut down};     
% %                 % Specification of lines between nodes specified above
% %                 % with aditional nodes for description 
% %                 \draw[->]             (start) -- (onCreateBlock);
% %                 \draw[->]     (onCreateBlock) -- (onStartBlock);
% %                 \draw[->]      (onStartBlock) -- (onResumeBlock);
% %                 \draw[->]      (activityRuns) -- node[text width=4cm] {Another activity comes in front of the activity} (onPauseBlock);
% %                 \draw[->]      (onPauseBlock) -- node {The activity is no longer visible} (onStopBlock);
% %                 \draw[->]       (onStopBlock) -- node {The activity is shut down by user or system} (onDestroyBlock);
% %                 \draw[->]    (onRestartBlock) -- (onStartBlock);
% %                 \draw[->]       (onStopBlock) -| node[yshift=1.25cm, text width=3cm]
% %                                                 {The activity comes to the foreground}
% %                                                 (onRestartBlock);
% %                 \draw[->]    (onDestroyBlock) -- (ActivityDestroyed);
% %                 \draw[->]      (onPauseBlock) -| node(priorityXMemory)
% %                                                 {higher priority $\rightarrow$ more memory}
% %                                                 (ActivityEnds);
% %                 \draw           (onStopBlock) -| (priorityXMemory);
% %                 \draw[->]     (ActivityEnds)  |- node [yshift=-2cm, text width=3.1cm]
% %                                                     {User navigates back to the activity}
% %                                                     (onCreateBlock);
% %                 \draw[->] (onPauseBlock.east) -- ++(2.6,0) -- ++(0,2) -- ++(0,2) --                
% %                     node[xshift=1.2cm,yshift=-1.5cm, text width=2.5cm]
% %                     {The activity comes to the foreground}(onResumeBlock.east);
% %             \end{tikzpicture}
% %         \column{.05\textwidth}
% %         \column{0.4\textwidth}
% %             ~\hfill
% %             \vbox{}\vskip-1.4ex%
% %             \begin{beamercolorbox}[sep=8em, colsep*=18pt, wd=\textwidth,ht=\paperheight]{title page header}
% %                 \begin{center}
% %                     \textbf{\huge{VISÃO}}\par
% %                     \vspace*{0.3cm}
% %                     \textbf{\huge{FUTURA}}
% %                 \end{center}
% %             \end{beamercolorbox}%
% %     \end{columns}
  
% %  %*----------- notes
% %     \note[item]{Notes can help you to remember important information. Turn on the notes option.}
% % \end{frame}
% %-
% %*----------- SLIDE -------------------------------------------------------------
% \begin{frame}{}
%     %\transdissolve[duration=0.5]
%     %\hspace*{-1cm}
    
%     \begin{columns}
%         %\column{.01\textwidth}
%         \column{.6\textwidth}
%         \begin{tikzpicture}[font=\sffamily, every matrix/.style={ampersand replacement=\&,column sep=1cm,row sep=1cm}, source/.style={draw,thick,rounded corners,fill=yellow!20,inner sep=.3cm},process/.style={draw,thick,circle,fill=blue!20}, sink/.style={source,fill=green!20},datastore/.style={draw,very thick,shape=datastore,inner sep=.3cm}, dots/.style={gray,scale=2}, to/.style={->,>=stealth',shorten >=1pt,semithick,font=\sffamily\footnotesize}, every node/.style={align=center}]
          
%             % Position the nodes using a matrix layout
%             \matrix{
%               \node[source] (hisparcbox) {electronics};
%                 \& \node[process] (daq) {DAQ}; \& \\
%                 \& \node[datastore] (buffer) {buffer}; \& \\
%                 \node[datastore] (storage) {storage};
%                 \& \node[process] (monitor) {monitor};
%                 \& \node[sink] (datastore) {datastore}; \\
%             };
          
%             % Draw the arrows between the nodes and label them.
%             \draw[to] (hisparcbox) -- node[midway,above] {raw events} node[midway,below] {level 0} (daq);
%             \draw[to] (daq) -- node[midway,right] {raw event data\\level 1} (buffer);
%             \draw[to] (buffer) -- node[midway,right] {raw event data\\level 1} (monitor);
%             \draw[to] (monitor) to[bend right=50] node[midway,above] {events} node[midway,below] {level 1} (storage);
%             \draw[to] (storage) to[bend right=50] node[midway,above] {events} node[midway,below] {level 1} (monitor);
%             \draw[to] (monitor) -- node[midway,above] {events} node[midway,below] {level 1} (datastore);
%           \end{tikzpicture}
                
        
%         \column{.05\textwidth}
%         \column{0.4\textwidth}
%             ~\hfill
%             \vbox{}\vskip-1.4ex%
%             \begin{beamercolorbox}[sep=8em, colsep*=18pt, wd=\textwidth,ht=\paperheight]{title page header}
%                 \begin{center}
%                     \textbf{\huge{VISÃO}}\par
%                     \vspace*{0.3cm}
%                     \textbf{\huge{FUTURA}}
%                 \end{center}
%             \end{beamercolorbox}%
%     \end{columns}
  
%  %*----------- notes
%     \note[item]{Notes can help you to remember important information. Turn on the notes option.}
% \end{frame}
% %-